\begin{appendices}
\crefalias{chapter}{appsec}
\chapter{Heat equation derivation}
\label{app:heatderive}

To derive the heat equation we consider the conversation of energy in a volume R, with a flux out, $\phi(x,y,z,t)$, and unit outer normal $\mathbf{\hat{n}}$. We need just the normal component of $\phi$ 
: $\phi \cdot \mathbf{\hat{n}}$.
\medskip

The rate of change of heat inside the volume R is equal to the heat generated inside the volume R plus the heat flowing in/out of the boundary surface:

\begin{equation}
\parbox{70pt}{Rate of change of heat energy}\ =\ \parbox{60pt}{Rate of heat generation in R} +\ \parbox{105pt}{Rate of heat energy flowing through boundary surface}
\label{eqn:word}
\end{equation}

\medskip
The total heat energy is:

\begin{equation}
e(x,y,z,t)=c(x,y,z)\cdot \rho(x,y,z)\cdot T(x,y,z,t)
\end{equation} 

and therefore the rate of change of heat energy is

\begin{equation}
\frac{d}{dt} \iiint\limits_{R} e\ dV= \frac{d}{dt}\iiint\limits_{R} c\rho T\ dV
\label{eqn:rotenergy}
\end{equation}

We denote the heat generated inside the volume R as $Q(x,y,z,t)$:

\begin{equation}
\iiint\limits_{R} Q\ dV
\label{eqn:rotheatgen}
\end{equation}

and the rate of heat energy flowing through the boundary surface is:

\begin{equation}
-\iint \limits_{\partial R} \phi\cdot \mathbf{\hat{n}}\ dS\footnote[3]{This is negative as outward flow $\phi$ is positive, but the flow would result in a reduction of energy.}
\label{eqn:rotheatloss}
\end{equation}

Substituting \cref{eqn:rotenergy,eqn:rotheatgen,eqn:rotheatloss} into \cref{eqn:word}, yields:

\begin{equation}
\frac{\partial}{\partial t} \iiint\limits_{R} c\rho T\ dV = -\iiint \limits_{R} \phi\cdot \mathbf{\hat{n}}\ dV +  \iiint\limits_{R} Q\ dV
\end{equation}

Using the divergence theorem, and simplifying gives: 

%\begin{equation}
%\iint \limits_{\partial R} \phi\cdot \mathbf{\hat{n}}\ dS = \iiint \limits_{R} \nabla\cdot \phi\ dV
%\end{equation}

\begin{equation}
\frac{\partial}{\partial t} \iiint\limits_{R} c\rho T\ dV = -\iiint \limits_{R} \nabla\cdot \phi\ dV +  \iiint\limits_{R} Q\ dV
\end{equation}

\begin{equation}
\iiint\limits_{R} \left[ c\rho \frac{\partial}{\partial t} T + \nabla\cdot \phi - Q\right] dV = 0
\end{equation}

Which holds for an arbitrary R, thus:

\begin{equation}
c\rho\frac{\partial}{\partial t}T = - \nabla \cdot \phi + Q
\label{eqn:heatpreq}
\end{equation}



Using Fourier's law of heat conduction, which states that the local heat flux density, $\phi$, is proportional to the negative local temperature gradient. The proportionality constant being equal to the thermal conductivity, $\kappa$:

\begin{equation}
\phi(x,y,z,t)=\kappa(x,y,z)\nabla T(x,y,z,t)
\label{eqn:fourier}
\end{equation}

Substituting \cref{eqn:fourier} into \cref{eqn:heatpreq} yields the heat equation:

\begin{equation}
c\rho\frac{\partial}{\partial t}T = \nabla\cdot (\kappa\nabla T) + Q
\end{equation}

Which can be simplified into the homogeneous medium heat equation with the following assumptions: Q=0 and $\kappa,\ \rho,\ and\ c$ are constant, and $\alpha=\tfrac{\kappa}{c\rho}$

\begin{equation}
\frac{\partial T}{\partial t} = \alpha \nabla^2 T
\end{equation}
\end{appendices}
