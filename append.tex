\begin{appendices}
\chapter{Heat equation derivation}

To derive the heat equation consider the conversation of energy

rate of change of heat energy = rate of heat energy flowing through boundary surface + rate of heat generation in R


\begin{equation}
e(x,y,z,t)=c(x,y,z)\cdot \rho(x,y,z)\cdot T(x,y,z,t)
\end{equation} 

\begin{equation}
\parbox{70pt}{rate of change of heat energy} =\frac{d}{dt} \iiint\limits_{R} e\ dV= \frac{d}{dt}\iiint\limits_{R} c\rho T\ dV
\end{equation}


\begin{equation}
\parbox{100pt}{rate of heat generation in R} = \iiint\limits_{R} Q\ dV
\end{equation}

\begin{equation}
\parbox{90pt}{rate of heat energy flowing through boundary surface} = -\iint \limits_{\partial R} \phi\cdot \bold{\hat{n}}\ dS
\end{equation}

\begin{equation}
\frac{\partial}{\partial t} \iiint\limits_{R} c\rho T\ dV = -\iiint \limits_{R} \phi\cdot \bold{\hat{n}}\ dV +  \iiint\limits_{R} Q\ dV
\end{equation}

div theorem 

\begin{equation}
\iint \limits_{\partial R} \phi\cdot \bold{\hat{n}}\ dS = \iiint \limits_{R} \nabla\cdot \phi\ dV
\end{equation}

\begin{equation}
\frac{\partial}{\partial t} \iiint\limits_{R} c\rho T\ dV = -\iiint \limits_{R} \nabla\cdot \phi\ dV +  \iiint\limits_{R} Q\ dV
\end{equation}

\begin{equation}
\iiint\limits_{R} \left[ c\rho \frac{\partial}{\partial t} T + \nabla\cdot \phi - Q\right] dV = 0
\end{equation}

holds for arbitrary R.

\begin{equation}
c\rho\frac{\partial}{\partial t}T = - \nabla \cdot \phi + Q
\end{equation}

\begin{equation}
\phi(x,y,z,t)=\kappa(x,y,z)\nabla T(x,y,z,t)
\end{equation}

\begin{equation}
c\rho\frac{\partial}{\partial t}T = \nabla\cdot (\kappa\nabla T) + Q
\end{equation}

Q=0 and $\kappa,\ \rho,\ and\ c$ are constant

\begin{equation}
\frac{\partial T}{\partial t} = \alpha \nabla^2 T
\end{equation}

\end{appendices}